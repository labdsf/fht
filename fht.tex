\documentclass{article}

\usepackage{amsmath}

\usepackage{tikz}
\usetikzlibrary{calc,angles,quotes}

\begin{document}

\section{Camera matrix}

\begin{equation}
C = K R_x R_y R_z,
\end{equation}
where $K$ is a camera calibration matrix.

\subsection{Camera calibration matrix}

\begin{equation}
K = \begin{bmatrix}
f & 0 & c_x \\
0 & f & c_y \\
0 & 0 & 1
\end{bmatrix}
\end{equation}

\subsection{Rotation matrices}

\begin{subequations}
\begin{equation}
R_x = \begin{bmatrix}
1 & 0 & 0 \\
0 & \cos(\alpha_x) & -sin(\alpha_x) \\
0 & \sin(\alpha_x) & \cos(\alpha_x)
\end{bmatrix}
\end{equation}

\begin{equation}
R_y = \begin{bmatrix}
\cos(\alpha_y) & 0 & \sin(\alpha_y) \\
0 & 1 & 0 \\
-\sin(\alpha_y) & 0 & \cos(\alpha_y)
\end{bmatrix}
\end{equation}

\begin{equation}
R_z = \begin{bmatrix}
\cos(\alpha_z) & -\sin(\alpha_z) & 0 \\
\sin(\alpha_z) & \cos(\alpha_z) & 0 \\
0 & 0 & 1
\end{bmatrix}
\end{equation}
\end{subequations}

\section{Vanishing points}

\subsection{Vanishing point detection}

Vanishing point is detected using two Hough transforms in sequence as proposed in \cite{nikolaev2008hough} and later reinvented in \cite{chen2010new}.
In between first and second transforms image gradient is calculated.

\subsection{Vanishing point coordinates}

Vanishing point matrix is calculated as follows \cite{gallagher2005using}:
\begin{equation}
V = K R_z R_x R_y
\end{equation}

First, the camera is rotated around $y$ axis, then around $x$ axis, and then tilted around $z$ axis.

We are only interested in vertical vanishing point:
\begin{equation}
v_y = V \begin{bmatrix}
0 \\
1 \\
0
\end{bmatrix} =
\begin{bmatrix}
\sin(\alpha_x) c_x - \cos(\alpha_x) \sin(\alpha_z) f \\
\cos(\alpha_x) \cos(\alpha_z) f + \sin(\alpha_x) c_y \\
\sin(\alpha_x)
\end{bmatrix}
\end{equation}

Note that coordinates of this vanishing point don't depend on $\alpha_y$.
For this reason we have to assume $\alpha_y = 0$.

Now, we have to undo two rotations.
In order to do that, we have to determine $f$, $\alpha_x$ and $\alpha_z$.

$\alpha_z$ is an angle between the $y$ axis and vanishing point position vector in the image plane.
In the code it is calculated with \texttt{atan2} function.

Now, we set $\alpha_z = 0$ and proceed to determine $f$ and $\alpha_x$.
Because the camera is not tilted, vanishing point lies on the $y$ axis of the image plain.
It is impossible to find two numbers, $f$ and $\alpha_x$, from one number.
We have to make an assumption that bottom of the image corresponds to the horizon.

\begin{tikzpicture}[
scale=7,
axis/.style={very thick, ->},
]
\draw[axis] (-0.1,0)  -- (1.1,0) node(xline)[right] {$Z$};
\draw[axis] (0,-0.1) -- (0,1.1) node(yline)[above] {$Y$};

\draw (0.5,0) -- (1.0,1.0);
\draw (1.0,0) -- (1.0,1.0);
\draw (1,1) -- (1.0,1.0);

\coordinate (A) at (0.5,0);
\coordinate (B) at (1,1);
\coordinate (P) at (1,0);
\coordinate (Q) at ($(A)!(P)!(B)$);

\draw (Q) -- node[above] {$f$} (P);
\path (Q) -- node[left] {$c_y$} (A);
\path (Q) -- node[left] {$v_y$} (B);

\path pic["$\alpha_x$", draw, angle eccentricity = 1.8] {angle = A--B--P};
\path pic["$\alpha_x$", draw, angle eccentricity = 1.8] {angle = Q--P--A};

\end{tikzpicture}

\begin{equation}
v_y = f \cot(\alpha_x)
\end{equation}
\begin{equation}
f = c_y \cot(\alpha_x)
\end{equation}

Using these two equations, we can find that
\begin{equation}
\tan(\alpha_x) = \sqrt{\frac{c_y}{v_y}}
\end{equation}

\bibliographystyle{unsrt}
\bibliography{fht}

\end{document}
